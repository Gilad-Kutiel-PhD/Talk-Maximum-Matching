\begin{enumerate}
\item
ראינו שכדי למצוא שידוך מקסימום מספיק למצוא מסלול משפר.
\item
הצגנו אלגוריתם פולינומי שמוצא שידוך מקסימום בגרף כללי. 
\item
האלגוריתם מתבסס על האבחנה שניתן לכווץ מעגלים אי זוגיים מבלי לשנות את 
(אי) הימצאותו של מסלול משפר/
\end{enumerate}

האלגוריתם המקורי של אדמונדס מ-1965 רץ בזמן 
$O(n^2m)$.
הצליחו לשפר מאז את זמן הריצה (רשימה חלקית):
\begin{enumerate}
\item $O(n^3)$ Balinski [1969]
\item $O(n^{2.5})$ Even and Kariv [1975]
\item $O(n^{0.5} m)$ Micali and Vazirani [1980]
\end{enumerate}

ב-2012 התפרסם מאמר בשם 
\textenglish{A Simplification of the MV Matching Algorithm and its Proof}
שמציג פישוט והוכחה לאלגוריתם מ-1980. המאמר מכיל 38 עמודים.
