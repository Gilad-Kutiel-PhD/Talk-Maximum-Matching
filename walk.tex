בהינתן גרף 
$G = (V, E)$
ושידוך $M$ נבנה את הגרף המכוון הבא:
$D = (V, A)$
\\
כאשר 
$A = \{(u, v) : \exists w, \{u, w\} \notin M, \{w, v\} \in M\}$.
\begin{figure}[h]
\centering
\begin{tikzpicture}[every node/.style={default node}, very thick]
\def\r{2}
\def\a{60}

\node[free](1) at (2,2) {1};
\node[free](2) at (0,0) {2};
\node[neighbor](3) at (2,0) {3};
\node(4) at (4,0) {4};
\node(5) at (6,0) {5};

\node(6) at (5,1.5) {6};
\node(7) at (6,3) {7};
\node[neighbor](8) at (4,3) {8};
\node[free](9) at (2,3) {9};

\begin{scope}[opacity=0.4]
\draw[notm] (1) -- (3);
\draw[notm] (2) -- (3);
\draw[m] (3) -- (4);
\draw[notm] (4) -- (5);
\draw[m] (5) -- (6);
\draw[notm] (6) -- (4);
\draw[notm] (6) -- (7);
\draw[m] (7) -- (8);
\draw[notm] (8) -- (9);
\end{scope}

\begin{scope}[->]
\draw (1) to[bend left] (4);
\draw (2) to[bend right] (4);
\draw (5) to[bend left] (3);
\draw (6) to[bend right] (3);
\draw (4) to[bend left] (6);
\draw (4) to[bend right] (5);
\draw (6) to[] (8);
\draw (9) to[bend left] (7);
\draw (7) to[bend left] (5);
\end{scope}
\end{tikzpicture}

\end{figure}
נסמן ב-%
$X \subseteq V$
את קבוצת הצמתים הפנויים.
\begin{claim}
מסלול משפר ב-$G$ הוא מסלול מכוון ב-$D$ מצומת ב-$X$ לצומת ב-%
$N(X)$.
\end{claim}
האם סיימנו ? האם כל מסלול מכוון ב-$D$ מצומת ב-$X$ לצומת ב-%
$N(X)$
הוא מסלול משפר ב-$G$ ?
\begin{claim}
מסלול מכוון קצר ביותר ב-$D$ מצומת ב-$X$ לצומת ב-%
$N(X)$
משרה טיול ב-$G$ שאינו מכיל מעגלים זוגיים.
\end{claim}

\begin{proof}
על ידי ציור. 
נניח בשלילה קיים מעגל זוגי, האם הקשתות שמסומנות ב-? הן חלק מהשידוך או לא ?
\end{proof}

\begin{figure}[h]
\centering
\begin{tikzpicture}[every node/.style={default node}, very thick]
\node[](1) at (0,0) {};
\node[](2) at (2,-2) {};
\node[](3) at (2,0) {};
\node[](4) at (4,0) {};
\node[](5) at (2,2) {};

\draw[] (1) -- (3) node[label above]{?};
\draw[] (2) -- (3) node[label above]{?};
\draw[m] (3) -- (4);
\draw[notm] (3) -- (5);
\draw[path] 
(4)		to[out=0, in=-45] 
(5,3)	to[out=135, in=90] 
(5);
\end{tikzpicture}

\end{figure}
