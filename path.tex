באופן לא פורמלי,
\emph{כיווץ}
של קשת מתבצע על ידי איחוד של שני קצוות הקשת לצומת חדש וביטול קשתות מקבילות בגרף החדש.

באותו אופן אפשר להגדיר כיווץ של מעגל בגרף.
אם בנוסף נתון לנו שידוך אז כאשר נבטל קשתות מקבילות נשאיר את קשת השידוך במידה וקיימת כזאת.

בהינתן גרף $G$, שידוך $M$ וטיול משפר עם מעגל (אי זוגי) ניצור גרף חדש, 
$G'$
על ידי כיווץ המעגל.
\begin{claim}
קיים ב-$G$ מסלול משפר אמ"מ קיים ב-
$G'$
מסלול משפר.
בנוסף, בהינתן מסלול משפר ב-
$G'$
ניתן למצוא ביעילות מסלול משפר ב-$G$.
\end{claim}

\begin{proof}
כיוון ראשון, נניח שב-%
$G'$
קיים מסלול משפר שעובר במעגל שכווץ, 
קל להשתכנע שב-$G$ קיים מסלול משפר שכולל את חציו העליון של המעגל.

\begin{figure}[h]
\centering
\begin{tikzpicture}[every node/.style={default node}, very thick]
\node[](-1) at (0,4) {};
\node[](0) at (0,2) {};
\node[](1) at (0,0) {};

\draw[notm] (-1) -- (0);
\draw[path] (0) -- (1);

\begin{scope}[opacity=.5]
\node[](2) at (2,0) {};
\node[](3) at (3,-1.5) {};
\node[](4) at (3,1.5) {};

\node[](5) at (6,0) {};
\node[](6) at (5,-1.5) {};
\node[](7) at (5,1.5) {};

\draw[m] (1) -- (2);
\draw[notm] (2) -- (3);
\draw[notm] (2) -- (4);

\draw[path] (4) -- (7);
\draw[path] (3) -- (6);

\draw[m] (7) -- (5);
\draw[notm] (6) -- (5);
\end{scope}

\node[](8) at(8,0) {};
\node[](9) at(8,2) {};
\node[](10) at(8,4) {};

\draw[notm] (5) -- (8); 
\draw[path] (8) -- (9);
\draw[notm] (9) -- (10); 

\draw[dashed]
(2.west) to[out=90, in=180] 
(4.north) to[]
(7.north) to[out=0, in=90]
(5.east) to[out=-90, in=0] 
(6.south) to[]
(3.south) to[out=180, in=-90]
(2.west)
;
\end{tikzpicture}


\end{figure}
כיוון שני, נניח שקיים טיול שמכיל מעגל אי זוגי, ובנוסף קיים ב-$G$ מסלול משפר.
אז קל להשתכנע שהחלק הראשון של הטיול בשילוב החלק השני של המסלול המשפר מהווה מסלול משפר ב-%
$G'$.
\begin{figure}[h]
\centering
\begin{tikzpicture}[every node/.style={default node}, very thick]
\node[](-1) at (0,4) {};
\node[](0) at (0,2) {};
\node[](1) at (0,0) {};

\draw[notm] (-1) -- (0);
\draw[path] (0) -- (1);

\begin{scope}[opacity=.5]
\node[](2) at (2,0) {};
\node[](3) at (3,-1.5) {};
\node[](4) at (3,1.5) {};

\node(13) at(4.5, 1.5) {};

\node[](5) at (7,0) {};
\node[](6) at (6,-1.5) {};
\node[](7) at (6,1.5) {};

\draw[m] (1) -- (2);
\draw[notm] (2) -- (3);
\draw[notm] (2) -- (4);


\draw[path] (4) -- (13);
\draw[path] (13) -- (7);
\draw[path] (3) -- (6);

\draw[m] (7) -- (5);
\draw[notm] (6) -- (5);
\end{scope}

\node[](8) at(9,0) {};
\node[](9) at(9,2) {};
\node[](10) at(9,4) {};

\draw[notm] (5) -- (8); 
\draw[path] (8) -- (9);
\draw[notm] (9) -- (10); 

\draw[dashed]
(2.west) to[out=90, in=180] 
(4.north) to[]
(7.north) to[out=0, in=90]
(5.east) to[out=-90, in=0] 
(6.south) to[]
(3.south) to[out=180, in=-90]
(2.west)
;

\node(11) at (0, -2) {};
\node(12) at (0, -4) {};

\draw[path] (1) -- (11);
\draw[notm] (11) -- (12);

\node[](14) at (6,4) {};
\node[](15) at (4.5,4) {};

\draw[notm] (14) -- (15);
\draw[path] (15) -- (13);

\draw[->, orange]
($(14) + (0,-.5)$)	to[out=180, in=45]
($(15) + (.5,-.7)$)	to[out=-135, in=135]
($(13) + (.5,.7)$)	to[out=-45, in=135]
($(7) + (1,.2)$) 	to[out=-45, in=135]
($(5) + (.5,.5)$)	to[out=-45, in=-135]
($(8) + (-.75,.5)$)	to[out=45, in=-90]
($(9) + (-.5,0)$)	to[out=90, in=-90]
($(10) + (-.5,0)$)
;

\draw[->, orange]
($(-1) + (.5,0)$)	to[]
($(0) + (.5,0)$)	to[out=-90, in=135]
($(1) + (.7,.5)$)	to[out=-45, in=-135]
($(2) + (0,.7)$) 	to[out=45, in=180]
($(4) + (0,-.4)$)	to[out=0, in=0]
($(13) + (0,-.4)$)	to[out=0, in=135]
($(7) + (0,-.7)$)	to[out=-45, in=90]
($(5) + (-.5,0)$)	to[out=-90, in=30]
($(6) + (0,.5)$)	to[out=210, in=-20]
($(3) + (0,.5)$)	to[out=160, in=-20]
($(2) + (0,-.5)$)	to[out=160, in=45]
($(1) + (.5,-.5)$)	to[out=-135, in=90]
($(11) + (.5,0)$)	to[out=-90, in=90]
($(12) + (.5,0)$)
;

\end{tikzpicture}


\end{figure}

\end{proof}

