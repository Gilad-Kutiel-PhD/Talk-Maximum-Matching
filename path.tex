באופן לא פורמלי,
\emph{כיווץ}
של קשת מתבצע על ידי איחוד של שני קצוות הקשת לצומת חדש וביטול קשתות מקבילות בגרף החדש.

באותו אופן אפשר להגדיר כיווץ של מעגל בגרף.
אם בנוסף נתון לנו שידוך אז כאשר נבטל קשתות מקבילות נשאיר את קשת השידוך במידה וקיימת כזאת.

בהינתן גרף $G$, שידוך $M$ וטיול משפר עם מעגל (אי זוגי) ניצור גרף חדש, 
$G'$
על ידי כיווץ המעגל.
\begin{claim}
קיים ב-$G$ מסלול משפר אמ"מ קיים ב-
$G'$
מסלול משפר.
בנוסף, בהינתן מסלול משפר ב-
$G'$
ניתן למצוא ביעילות מסלול משפר ב-$G$.
\end{claim}
