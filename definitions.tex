\begin{definition}[שידוך]
שידוך בגרף לא מכוון
$G = (V, E)$
הוא תת קבוצה של קשתות 
$M \subseteq E$
כך שכל שתי קשתות
$e_1, e_2 \in M$
מתקיים ש-
$e_1 \cap e_2 = \emptyset$.
\end{definition}

\begin{definition}[שידוך מקסימום]
שידוך מקסימום הוא שידוך 
$M \subseteq E$
כך שלכל שידוך אחר
$M' \subseteq E$
מתקיים
$|M| \geq |M'|$
\end{definition}

\textbf{הערה:}
שידוך מקסימום אינו יחיד בהכרח.

\begin{definition}[שידוך מקסימלי]
שידוך מקסימלי הוא שידוך 
$M \subseteq E$
כך שלכל קשת 
$e \in E \setminus M$
קיימת קשת 
$e' \in M$
כך ש-
$e \cap e' \neq \emptyset$
\end{definition}

\textbf{הערה:}
קל למצוא שידוך מקסימלי, למשל באופן חמדני.

