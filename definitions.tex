\begin{definition}[שידוך]
שידוך בגרף לא מכוון
$G = (V, E)$
הוא תת קבוצה של קשתות 
$M \subseteq E$
כך שכל שתי קשתות
$e_1, e_2 \in M$
מתקיים ש-
$e_1 \cap e_2 = \emptyset$.
\end{definition}
\textbf{הערה:}
מתייחסים לשידוך גם בתור 
\emph{קבוצה בלתי תלויה}
של קשתות.

\begin{definition}[שידוך מקסימום]
שידוך מקסימום הוא שידוך 
$M \subseteq E$
כך שלכל שידוך אחר
$M' \subseteq E$
מתקיים
$|M| \geq |M'|$
\end{definition}

\textbf{הערה:}
שידוך מקסימום אינו יחיד בהכרח.

\begin{definition}[שידוך מקסימלי]
שידוך מקסימלי הוא שידוך 
$M \subseteq E$
כך שלכל קשת 
$e \in E \setminus M$
קיימת קשת 
$e' \in M$
כך ש-
$e \cap e' \neq \emptyset$
\end{definition}

\textbf{הערה:}
קל למצוא שידוך מקסימלי, למשל באופן חמדני.

\begin{definition}[שידוך מושלם]
שידוך מושלם הוא שידוך בגודל 
$n/2$.
\end{definition}
\textbf{הערה:}
לא בכל גרף קיים שידוך מושלם.

ההגדרות הבאות הן בהינתן שידוך $M$.
\begin{definition}[צומת תפוס]
צומת $u$ יקרא
\emph{תפוס}
אם קיים צומת $v$ כך ש-
$\{u, v\} \in M$.
\end{definition}

\begin{definition}[צומת פנוי]
צומת $u$ יקרא
\emph{פנוי}
אם הוא לא תפוס.
$\{u, v\} \in M$.
\end{definition}

\begin{definition}[מסלול מתחלף]
מסלול (פשוט) 
$(v_1, \ldots, v_l)$
יקרא
\emph{מתחלף}
אם לכל 
$1 \leq i < l$
מתקיים ש-
$|\{e_i, e_{i + 1}}\} \cap M| = 1$
כאשר
$e_i = \{v_i, v_{i + 1}\}$.
\end{definition}

\begin{definition}[מסלול משפר]
מסלול משפר הוא מסלול מתחלף ששני קצותיו הם צמתים פנויים.
\end{definition}

\begin{figure}[h]
\centering
\begin{tikzpicture}[every node/.style={default node}, very thick]
\def\r{2}
\def\a{60}

\node(1) at (2,2) {1};
\node(2) at (0,0) {2};
\node(3) at (2,0) {3};
\node(4) at (4,0) {4};
\node(5) at (6,0) {5};

\node(6) at (5,1.5) {6};
\node(7) at (6,3) {7};
\node(8) at (4,3) {8};
\node(9) at (2,3) {9};

\draw[notm] (1) -- (3);
\draw[notm] (2) -- (3);
\draw[m] (3) -- (4);
\draw[notm] (4) -- (5);
\draw[m] (5) -- (6);
\draw[notm] (6) -- (4);
\draw[notm] (6) -- (7);
\draw[m] (7) -- (8);
\draw[notm] (8) -- (9);
\end{tikzpicture}

\end{figure}
