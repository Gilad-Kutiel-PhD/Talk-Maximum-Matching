\begin{definition}[שידוך]
שידוך בגרף לא מכוון
$G = (V, E)$
הוא תת קבוצה של קשתות 
$M \subseteq E$
כך שכל שתי קשתות
$e_1, e_2 \in M$
מתקיים ש-
$e_1 \cap e_2 = \emptyset$.
\end{definition}

\begin{definition}[שידוך מקסימום]
שידוך מקסימום הוא שידוך 
$M \subseteq E$
כך שלכל שידוך אחר
$M' \subseteq E$
מתקיים
$|M| \geq |M'|$
\end{definition}

\textbf{הערה:}
שידוך מקסימום אינו יחיד בהכרח.

\begin{definition}[שידוך מקסימלי]
שידוך מקסימלי הוא שידוך 
$M \subseteq E$
כך שלכל קשת 
$e \in E \setminus M$
קיימת קשת 
$e' \in M$
כך ש-
$e \cap e' \neq \emptyset$
\end{definition}

\textbf{הערה:}
קל למצוא שידוך מקסימלי, למשל באופן חמדני.

\begin{definition}[שידוך מושלם]
שידוך מושלם הוא שידוך בגודל 
$n/2$.
\end{definition}

ההגדרות הבאות הן בהינתן שידוך $M$.
\begin{definition}[צומת תפוס]
צומת $u$ יקרא
\emph{תפוס}
אם קיים צומת $v$ כך ש-
$\{u, v\} \in M$.
\end{definition}

\begin{definition}[צומת פנוי]
צומת $u$ יקרא
\emph{פנוי}
אם הוא לא תפוס.
$\{u, v\} \in M$.
\end{definition}

\begin{definition}[מסלול מתחלף]
מסלול (פשוט) 
$(v_1, \ldots, v_l)$
יקרא
\emph{מתחלף}
אם לכל 
$1 \leq i < l$
מתקיים ש-
$|\{e_i, e_{i + 1}}\} \cap M| = 1$
כאשר
$e_i = \{v_i, v_{i + 1}\}$.
\end{definition}

