\begin{theorem}
שידוך $M$ הוא שידוך מקסימום אמ"מ לא קיים ב-$G$ מסלול משפר.
\end{theorem}

\begin{proof}
\textbf{כיוון ראשון:}
אם קיים מסלול משפר, $P$, אז 
$M \otimes P$
שידוך גדול יותר מ-$M$.
\\
\textbf{כיוון שני:}
אם $M$ לא שידוך מקסימום ו-$M'$ שידוך גדול ממנו אז בגרף 
$G' = (V, M \cup M')$
דרגת כל צומת היא 2 לכל היותר, כלומר כל רכיב קשיר הוא מעגל או מסלול.
במעגל, יש מספר קשתות זהה מ-$M$ ומ-$M'$.
מכיוון ש-$M'$ גדול יותר מ-$M$ חייב להיות מסלול עם קשת אחת יותר מ-$M'$, זה מסלול משפר.
\end{proof}

\textbf{מסקנה:}
כדי למצוא שידוך מקסימום בגרף, מספיק למצוא מסלול משפר.

בהינתן גרף 
$G = (V, E)$
ושידוך $M$ נבנה את הגרף המכוון הבא:
$D = (V, A)$
\\
כאשר 
$A = \{(u, v) : \exists w, \{u, w\} \notin M, \{w, v\} \in M\}$.
נסמן ב-%
$X \subseteq V$
את קבוצת הצמתים הפנויים.
\begin{claim}
מסלול משפר ב-$G$ הוא מסלול מכוון ב-$D$ מצומת ב-$X$ לצומת ב-%
$N(X)$.
\end{claim}
האם סיימנו ? האם כל מסלול מכוון ב-$D$ מצומת ב-$X$ לצומת ב-%
$N(X)$
הוא מסלול משפר ב-$G$ ?
\begin{claim}
מסלול מכוון ב-$D$ מצומת ב-$X$ לצומת ב-%
$N(X)$
אינו מכיל מעגלים זוגיים ב-$G$.
\end{claim}

